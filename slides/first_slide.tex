\documentclass[14pt]{beamer}

%% PACKAGE
%%
\usepackage[utf8]{inputenc}
\usepackage[T1]{fontenc}
\usepackage{lmodern}
\usepackage[portuguese]{babel}
\usepackage{amsmath}
\usepackage{amsfonts}
\usepackage{amssymb}
\usepackage{graphicx}
\usetheme{CambridgeUS}



\begin{document}
	\author{Helio M Andrade}
	\title{Estudo Slides LaTeX}
	%\subtitle{}
	\logo{ \includegraphics[height=2cm]{figuras/birb_logo}{\vspace{150pt}} }
	\institute{Udemy}
	%\date{}
	%\subject{}
	%\setbeamercovered{transparent}
	%\setbeamertemplate{navigation symbols}{}
	
	\begin{frame}[plain] % [plain] : retira bordas, usado em primeiro slide
		\maketitle % escreve título a partir das informações acima
	\end{frame}
	
	%%
	%%	ESTUDO SOBRE CRIAÇÃO DE FRAMES E BLOCOS
	%%
	\begin{frame}
		\frametitle{Primeiro Frame}
		Texto do frame em questão
		
		\begin{block}{Título do bloco}
			Bloco de texto aleatório
		\end{block}
	\end{frame}
	
	\begin{frame}
		\frametitle{Slide pausado}
		Texto do frame 2 em questão
		
		\begin{block}{Título do bloco}
			Bloco de texto aleatório 1 \pause
			
			Bloco de texto aleatório 2 \pause
			
			Bloco de texto aleatório 3 
		\end{block}
	\end{frame}

	\begin{frame}
		\frametitle{Slide de lista ordenada}
		Texto qualquer falando sobre algo
		\begin{enumerate}[<+->] %[<+->] + : na contagem de pausa o + acrescenta mais 1 
			\item Item bla 
			\item Item blabla 
			\item Item blablabla
		\end{enumerate}
	\end{frame}

	\begin{frame}
		\frametitle{Uso de uncover}
		\begin{block}{Título do bloco}
			\begin{itemize}
				\uncover<1->{\item Qualquer abobrinha qualquer:} %
				\uncover<3->
				{
					\begin{equation}
						ax + b = 0
					\end{equation}
				}
				
				\uncover<2->{\item Outra abobrinha qualquer:}
				\uncover<4->
				{
					\begin{equation}
						x^2 + bx + c = 0
					\end{equation}
				}
			\end{itemize}
		\end{block}
	\end{frame}

	\begin{frame}
		\frametitle{Uso de only}
		\begin{block}{Título do bloco}
			\begin{itemize}
				\uncover<1->{\item Comando only não reserva espaço:}  
				\only<3-> %\only{} não reserva espaço antes de exibir
				{
					\begin{equation}
					ax + b = 0
					\end{equation}
				}
				
				\uncover<2->{\item Outra abobrinha qualquer:}
				\only<4->
				{
					\begin{equation}
					x^2 + bx + c = 0
					\end{equation}
				}
			\end{itemize}
		\end{block}
	\end{frame}

	\begin{frame}
		\frametitle{Uso de <?->}
		\begin{block}{Título do bloco}
			\begin{itemize}
				\uncover<1-3>{\item Comando only não reserva espaço:}  % - : exibe elemento até pauxa X"
				\only<3> %\only{} não reserva espaço antes de exibir
				{
					\begin{equation}
					ax + b = 0
					\end{equation}
				}
				
				\uncover<2->{\item O item anterior foi ocultado na exibição da fórmula (pausa 4):}
				\only<4->
				{
					\begin{equation}
					x^2 + bx + c = 0
					\end{equation}
				}
			\end{itemize}
		\end{block}
	\end{frame}
	
\end{document}
	
	
